% !TeX root = ../main-presentation.tex
\section{Equational reasoning}

\begin{frame}
    \frametitle{Equality of circuits}

    When are two circuits equal?

    \wait

    When they have the same \alert{beahviour}.

    \wait

    \[
        \tikzfig{circuits/examples/demorgan-lhs} 
        \quad
        \tikzfig{circuits/examples/demorgan-rhs} 
    \]

    When they have the same \alert{stream function}.

    \wait

    But reasoning with streams is a \alert{pain}.
    
\end{frame}

\begin{frame}
    \frametitle{Equational reasoning}

    We want to reason \alert{equationally}.

\end{frame}

\begin{frame}
    \frametitle{Productivity}


    A closed circuit is \alert{productive} if it can be reduced to an \alert{instant value} and a \alert{delayed subcircuit}.

    \[
        \tikzfig{circuits/components/circuits/f-seq-closed}
        \quad
        =
        \quad    
        \tikzfig{circuits/productivity/productive}
    \]

\end{frame}

\begin{frame}
    \frametitle{Combinational equations}
    \setlength{\jot}{2em}
    \wait
    \begin{center}
        \[
            \tikzfig{circuits/axioms/gate-lhs}
            \quad=\quad
            \tikzfig{circuits/axioms/gate-rhs}  
            \wait
            \qquad
            \tikzfig{strings/cartesian/naturality-copy-lhs}
            \quad=\quad
            \tikzfig{strings/cartesian/naturality-copy-rhs}
        \]\[
            \wait
            \tikzfig{circuits/axioms/join-lhs}
            \quad=\quad
            \tikzfig{circuits/axioms/join-rhs}
            \wait
            \qquad
            \tikzfig{strings/cartesian/naturality-discard-lhs}
            \quad=\quad
            \tikzfig{strings/cartesian/naturality-discard-rhs}
            \wait
        \]\[
            \tikzfig{strings/structure/comonoid/unitality-r-lhs}
            \quad=\quad
            \tikzfig{strings/structure/comonoid/unitality-r-rhs}
            \quad=\quad
            \tikzfig{strings/structure/comonoid/unitality-l-lhs}
        \]
    \end{center}

    \wait
    These reduce any \alert{closed combinational circuit} \(\tikzfig{circuits/components/circuits/f-comb-applied}\) to \(\tikzfig{circuits/components/values/ws}\).

\end{frame}

\begin{frame}
    \frametitle{Sequential equations}

    \[
        \tikzfig{circuits/axioms/disconnect-lhs}
        \quad=\quad
        \tikzfig{circuits/axioms/disconnect-rhs}    
        \qquad\wait
        \tikzfig{circuits/axioms/streaming-lhs-verbose}
        \quad=\quad
        \tikzfig{circuits/axioms/streaming-rhs}    
    \]
\end{frame}

\begin{frame}
    \frametitle{Non delay-guarded feedback}

    How de we deal with something like this?

    \[
        \tikzfig{circuits/productivity/trand}   
    \]

    \wait

    We need a way to eliminate \alert{non delay-guarded feedback}.

    \[
        \tikzfig{circuits/components/circuits/f-seq-traced}  
    \]

\end{frame}

\begin{frame}
    \frametitle{Non delay-guarded feedback}

    \wait

    Our gates are \alert{monotonic}, so they must have a \alert{least fixed point}...
    
    \[f^i(\bot) = f^{i+1}(\bot)\]

    \wait

    Because the value set \(\values\) is finite, we can always find this fixpoint!    
    
\end{frame}

\begin{frame}
    \frametitle{Non delay-guarded feedback}

    \(
        \tikzfig{circuits/a4}    
    \)
    \quad
    In \(\values\), the length of the longest chain is \alert{3}...

    \wait

    \[
        \tikzfig{circuits/components/circuits/f-seq-traced}
        \quad=\quad
        \tikzfig{circuits/instant-feedback/concrete-unfolding}
    \]
    

\end{frame}

\begin{frame}
    \frametitle{Eliminating `instant' feedback}

    \begin{center}
        \begin{minipage}{0.25\textwidth}
            \tikzfig{circuits/productivity/trand} 
            \quad\(=\)
        \end{minipage}
        \begin{minipage}{0.4\textwidth}
            \only<1>{
                \begin{center}
                    \tikzfig{circuits/examples/trand/step-1}
                \end{center}
            } 
            \only<2>{
                \begin{center}
                    \tikzfig{circuits/examples/trand/step-2}
                \end{center}
            }
            \only<3>{
                \begin{center}
                    \tikzfig{circuits/examples/trand/step-3}
                \end{center}
            }
            \only<4>{
                \begin{center}
                    \tikzfig{circuits/examples/trand/step-4}
                \end{center}
            }
        \end{minipage}
    \end{center}
\end{frame}

\begin{frame}
    \frametitle{Productivity}

    Now \alert{any} closed circuit is productive!


    \only<1>{
        \begin{center}
            \tikzfig{circuits/examples/reasoning/unfolding-dg/step-0}
        \end{center}
    }
    \only<2>{
        \begin{center}
            \tikzfig{circuits/examples/reasoning/unfolding-dg/step-0a}
        \end{center}
    }
    \only<3>{
        \begin{center}
            \tikzfig{circuits/examples/reasoning/unfolding-dg/step-1}
        \end{center}
    }
    \only<4>{
        \begin{center}
            \tikzfig{circuits/examples/reasoning/unfolding-dg/step-2}
        \end{center}
    }
    \only<5>{
        \begin{center}
            \tikzfig{circuits/examples/reasoning/unfolding-dg/step-3}
        \end{center}
    }
    \only<6>{
        \begin{center}
            \tikzfig{circuits/examples/reasoning/unfolding-dg/step-4}
        \end{center}
    }
    \only<7>{
        \begin{center}
            \tikzfig{circuits/examples/reasoning/unfolding-dg/step-5}
        \end{center}
    }
    \only<8>{
        \begin{center}
            \tikzfig{circuits/examples/reasoning/unfolding-dg/step-6}
        \end{center}
    }
    \only<9>{
        \begin{center}
            \tikzfig{circuits/examples/reasoning/unfolding-dg/step-7}
        \end{center}
    }
    \only<10>{
        \begin{center}
            \tikzfig{circuits/examples/reasoning/unfolding-dg/step-8}
        \end{center}
    }
    \only<11>{
        \begin{center}
            \tikzfig{circuits/examples/reasoning/unfolding-dg/step-9}
        \end{center}
    }
    \only<12>{
        \begin{center}
            \tikzfig{circuits/examples/reasoning/unfolding-dg/step-10}
        \end{center}
    }
    \only<13>{
        \begin{center}
            \tikzfig{circuits/examples/reasoning/unfolding-dg/step-11}
        \end{center}
    }
    \only<14>{
        \begin{center}
            \tikzfig{circuits/examples/reasoning/unfolding-dg/step-12}
        \end{center}
    }
    \only<15>{
        \begin{center}
            \tikzfig{circuits/examples/reasoning/unfolding-dg/step-13}
        \end{center}
    }
\end{frame}

\begin{frame}
    \frametitle{Open circuits}

    We still cannot necessarily translate between \alert{open} circuits with the same behaviour.

    \[
        \tikzfig{circuits/examples/demorgan-lhs} 
        \quad
        \tikzfig{circuits/examples/demorgan-rhs} 
    \]

    We need another equation.

\end{frame}

\begin{frame}
    \frametitle{Open circuits}

    When do these have the \alert{same stream}?

    \[
        \tikzfig{circuits/components/circuits/f-seq}
        \qquad
        \tikzfig{circuits/components/circuits/g-seq}
    \]

\end{frame}

\begin{frame}
    \frametitle{Open circuits}

    Put into \alert{global delay form}.

    \[
        \tikzfig{circuits/components/circuits/f-seq}
        \quad=\quad
        \tikzfig{circuits/full-abstraction/global-delay-f}
        \qquad
        \tikzfig{circuits/components/circuits/g-seq}
        \quad=\quad
        \tikzfig{circuits/full-abstraction/global-delay-g}
    \]
    
    Produces the \alert{state transition} and \alert{output} of the circuit.

    \pause

    \alert{Idea}: for all \alert{accessible states}, if the \alert{outputs} are equal then the \alert{original circuits} are equal under the equational theory.

    \tiny{(cf. Mealy machine bisimulation)}
\end{frame}

\begin{frame}
    \frametitle{Open circuits: example}

    \[
        \tikzfig{circuits/examples/simple-and}    
        \qquad
        \tikzfig{circuits/examples/simple-or-nots}
    \]

    \pause

    \[
        \tikzfig{circuits/examples/open-circuit-example/f-core}    
        \qquad
        \tikzfig{circuits/examples/open-circuit-example/g-core}
    \]

    Initial states are \((\tikzfig{circuits/components/values/true}, \tikzfig{circuits/components/values/true})\).

\end{frame}

\begin{frame}
    \frametitle{Open circuits: example}

    Check the \alert{initial output} and \alert{transition}...

    \begin{center}
        \begin{minipage}{0.45\textwidth}
            \only<1>{
                \begin{center}
                    \tikzfig{circuits/examples/open-circuit-example/init-f-0}
                \end{center}
            }
            \only<2>{
                \begin{center}
                    \tikzfig{circuits/examples/open-circuit-example/init-f-1}
                \end{center}
            }
            \only<3>{
                \begin{center}
                    \tikzfig{circuits/examples/open-circuit-example/init-f-2}
                \end{center}
            }
            \only<4->{
                \begin{center}
                    \tikzfig{circuits/examples/open-circuit-example/init-f-3}
                \end{center}
            }
        \end{minipage}
        \begin{minipage}{0.45\textwidth}
        \only<1-5>{
            \begin{center}
                \tikzfig{circuits/examples/open-circuit-example/init-g-0}
            \end{center}
        }
        \only<6>{
            \begin{center}
                \tikzfig{circuits/examples/open-circuit-example/init-g-1}
            \end{center}
        }
        \only<7>{
            \begin{center}
                \tikzfig{circuits/examples/open-circuit-example/init-g-2}
            \end{center}
        }
        \only<8>{
            \begin{center}
                \tikzfig{circuits/examples/open-circuit-example/init-g-3}
            \end{center}
        }
        \only<9>{
            \begin{center}
                \tikzfig{circuits/examples/open-circuit-example/init-g-4}
            \end{center}
        }
        \only<10->{
            \begin{center}
                \tikzfig{circuits/examples/open-circuit-example/init-g-5}
            \end{center}
        }
        \end{minipage}
    \end{center}

    \visible<11->{

        The \alert{next states} for each circuit are equal...

        \begin{center}
            \begin{minipage}{0.3\textwidth}
                \only<1-11>{
                    \begin{center}
                        \tikzfig{circuits/examples/open-circuit-example/next-f-0}
                    \end{center}
                }
                \only<12>{
                    \begin{center}
                        \tikzfig{circuits/examples/open-circuit-example/next-f-1}
                    \end{center}
                }
                \only<13->{
                    \begin{center}
                        \tikzfig{circuits/examples/open-circuit-example/next-f-2}
                    \end{center}
                }
            \end{minipage}
            \begin{minipage}{0.3\textwidth}
            \only<1-14>{
                \begin{center}
                    \tikzfig{circuits/examples/open-circuit-example/next-g-0}
                \end{center}
            }
            \only<15>{
                \begin{center}
                    \tikzfig{circuits/examples/open-circuit-example/next-g-1}
                \end{center}
            }
            \only<16>{
                \begin{center}
                    \tikzfig{circuits/examples/open-circuit-example/next-g-2}
                \end{center}
            }
            \only<17>{
                \begin{center}
                    \tikzfig{circuits/examples/open-circuit-example/next-g-3}
                \end{center}
            }
            \only<18>{
                \begin{center}
                    \tikzfig{circuits/examples/open-circuit-example/next-g-4}
                \end{center}
            }
            \only<19->{
                \begin{center}
                    \tikzfig{circuits/examples/open-circuit-example/next-g-5}
                \end{center}
            }
            \end{minipage}
        \end{center}
    }

    \visible<20>{
        So the original circuits are equal in the equational theory.
    }
\end{frame}

\begin{frame}
    \frametitle{Open circuits: example}

    

\end{frame}