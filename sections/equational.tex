% !TeX root = ../main-presentation.tex
\section{Equational reasoning}

\begin{frame}
    \frametitle{Equality of circuits}

    When are two circuits equal?
    \wait
    When they have the same \alert{behviour}

    \wait

    \[
        \tikzfig{circuits/examples/demorgan-lhs} 
        \quad
        \tikzfig{circuits/examples/demorgan-rhs} 
    \]

    \wait

    When they have the same \alert{stream function}

    \wait

    Reasoning with streams is a \alert{pain}.
    
\end{frame}

\begin{frame}
    \frametitle{Productivity}

    We want to reason \alert{equationally}: what equations do we need?

    \wait

    First goal: \alert{productivity}.

    \wait

    A closed circuit is \alert{productive} if it is equal to an \alert{instant value} and a \alert{delayed subcircuit} under the equational theory.

    \[
        \tikzfig{circuits/components/circuits/f-seq-closed}
        \quad
        =
        \quad    
        \tikzfig{circuits/productivity/productive}
    \]

\end{frame}

\begin{frame}
    \frametitle{Combinational equations}
    \setlength{\jot}{2em}
    \wait
    \begin{center}
        \[
            \tikzfig{circuits/axioms/gate-lhs}
            \quad=\quad
            \tikzfig{circuits/axioms/gate-rhs}  
            \wait
            \qquad
            \tikzfig{circuits/axioms/join-lhs}
            \quad=\quad
            \tikzfig{circuits/axioms/join-rhs}
            \wait
        \]
        \[
            \tikzfig{strings/cartesian/naturality-copy-lhs}
            \quad=\quad
            \tikzfig{strings/cartesian/naturality-copy-rhs}
            \qquad
            \wait
            \tikzfig{strings/cartesian/naturality-discard-lhs}
            \quad=\quad
            \tikzfig{strings/cartesian/naturality-discard-rhs}
            \wait
        \]
        \[
            \tikzfig{strings/structure/monoid/unitality-l-lhs}
            \quad=\quad
            \tikzfig{strings/structure/monoid/unitality-r-rhs}
            \quad=\quad
            \tikzfig{strings/structure/monoid/unitality-r-lhs}
            \qquad
            \tikzfig{strings/structure/comonoid/unitality-l-lhs}
            \quad=\quad
            \tikzfig{strings/structure/comonoid/unitality-r-rhs}
            \quad=\quad
            \tikzfig{strings/structure/comonoid/unitality-l-lhs}
        \]
    \end{center}

    \wait
    These reduce any \alert{closed combinational circuit} \(\tikzfig{circuits/components/circuits/f-comb-applied}\) to some \(\tikzfig{circuits/components/values/ws}\).

\end{frame}

\begin{frame}
    \frametitle{Sequential equations}
    \[
        \tikzfig{circuits/axioms/disconnect-lhs}
        \quad=\quad
        \tikzfig{circuits/axioms/disconnect-rhs}    
        \qquad\wait
        \tikzfig{circuits/axioms/streaming-lhs-verbose}
        \quad=\quad
        \tikzfig{circuits/axioms/streaming-rhs}    
    \]
\end{frame}

\begin{frame}
    \frametitle{Non delay-guarded feedback}

    How do we deal with something like this?

    \[
        \tikzfig{circuits/productivity/trand}   
    \]

    \wait

    We need a way to eliminate \alert{non delay-guarded feedback}.

    \[
        \tikzfig{circuits/components/circuits/f-seq-traced}  
    \]

\end{frame}

\begin{frame}
    \frametitle{Non delay-guarded feedback}

    \wait

    Our gates are \alert{monotonic}, so they must have a \alert{least fixed point}...

    \wait

    Because the value set \(\values\) is finite, we can always find this fixpoint!    
    
\end{frame}

\begin{frame}
    \frametitle{Non delay-guarded feedback}

    \(
        \tikzfig{circuits/a4}    
    \)
    \quad
    In \(\values\), the length of the longest chain is \alert{2}...

    \wait

    \[
        \tikzfig{circuits/components/circuits/f-comb-traced}
        \quad=\quad
        \tikzfig{circuits/instant-feedback/concrete-unfolding}
    \]
    

\end{frame}

\newcommand{\transition}{\Downarrow}

\NewDocumentCommand\animationframe{momm}{
    \only<#1>{
        \begin{minipage}{0.49\textwidth}
            \IfValueT{2}{
                \begin{center}
                    #2
                \end{center}
            }
        \end{minipage}
        \begin{minipage}{0.49\textwidth}
            \begin{center}
                #3

                \(\transition\)

                #4
            \end{center}
        \end{minipage}
    }
}

\begin{frame}
    \frametitle{Productivity}

    \only<1>{
        We want
        \begin{center}
            \tikzfig{circuits/components/circuits/f-seq-closed}
            \quad=\quad
            \tikzfig{circuits/productivity/productive}
        \end{center}
    }

    \animationframe{2}[
        Axioms of STMCs
    ]{
        \tikzfig{circuits/examples/reasoning/unfolding-dg/step-0}
    }{
        \tikzfig{circuits/examples/reasoning/unfolding-dg/step-0a}
    }
    \animationframe{3}[
        Eliminating `instant feedback'
    ]{
        \tikzfig{circuits/examples/reasoning/unfolding-dg/step-0a}
    }{
        \tikzfig{circuits/examples/reasoning/unfolding-dg/step-1}
    }
    \animationframe{4}[
        Axioms of STMCs
    ]{
        \tikzfig{circuits/examples/reasoning/unfolding-dg/step-1}
    }{
        \tikzfig{circuits/examples/reasoning/unfolding-dg/step-2}
    }
    \animationframe{5}[
        \tikzfig{strings/structure/comonoid/unitality-l-lhs}
        \(=\)
        \tikzfig{strings/structure/comonoid/unitality-l-rhs}
        \(=\)
        \tikzfig{strings/structure/comonoid/unitality-r-lhs}
    ]{
        \tikzfig{circuits/examples/reasoning/unfolding-dg/step-2}
    }{
        \tikzfig{circuits/examples/reasoning/unfolding-dg/step-3}
    }
    \animationframe{6}[
        \tikzfig{strings/cartesian/naturality-copy-lhs}
        \(=\)
        \tikzfig{strings/cartesian/naturality-copy-rhs}
    ]{
        \tikzfig{circuits/examples/reasoning/unfolding-dg/step-3}
    }{
        \tikzfig{circuits/examples/reasoning/unfolding-dg/step-4}
    }
    \animationframe{7}[
        \tikzfig{strings/cartesian/naturality-copy-lhs}
        \(=\)
        \tikzfig{strings/cartesian/naturality-copy-rhs}
    ]{
        \tikzfig{circuits/examples/reasoning/unfolding-dg/step-4}
    }{
        \tikzfig{circuits/examples/reasoning/unfolding-dg/step-5}
    }
    \animationframe{8}[
        Axioms of STMCs
    ]{
        \tikzfig{circuits/examples/reasoning/unfolding-dg/step-5}
    }{
        \tikzfig{circuits/examples/reasoning/unfolding-dg/step-6}
    }
    \animationframe{9}[
        Axioms of STMCs
    ]{
        \tikzfig{circuits/examples/reasoning/unfolding-dg/step-6}
    }{
        \tikzfig{circuits/examples/reasoning/unfolding-dg/step-7}
    }
    \animationframe{10}[
        \tikzfig{strings/cartesian/naturality-copy-lhs}
        \(=\)
        \tikzfig{strings/cartesian/naturality-copy-rhs}
    ]{
        \tikzfig{circuits/examples/reasoning/unfolding-dg/step-7}
    }{
        \tikzfig{circuits/examples/reasoning/unfolding-dg/step-8}
    }
    \animationframe{11}[
        \tikzfig{strings/structure/monoid/unitality-l-lhs}
        \(=\)
        \tikzfig{strings/structure/monoid/unitality-l-rhs}
        \(=\)
        \tikzfig{strings/structure/monoid/unitality-r-lhs}
    ]{
        \tikzfig{circuits/examples/reasoning/unfolding-dg/step-8}
    }{
        \tikzfig{circuits/examples/reasoning/unfolding-dg/step-9}
    }
    \animationframe{12}[
        \tikzfig{circuits/axioms/disconnect-rhs}
        \(=\)
        \tikzfig{circuits/axioms/disconnect-lhs}
    ]{
        \tikzfig{circuits/examples/reasoning/unfolding-dg/step-9}
    }{
        \tikzfig{circuits/examples/reasoning/unfolding-dg/step-10}
    }
    \animationframe{13}[
        \tikzfig{circuits/axioms/streaming-lhs-verbose}
        \(=\)
        \tikzfig{circuits/axioms/streaming-rhs}
    ]{
        \tikzfig{circuits/examples/reasoning/unfolding-dg/step-10}
    }{
        \tikzfig{circuits/examples/reasoning/unfolding-dg/step-11}
    }
    \animationframe{14}[
        Combinational circuit equations
    ]{
        \tikzfig{circuits/examples/reasoning/unfolding-dg/step-11}
    }{
        \tikzfig{circuits/examples/reasoning/unfolding-dg/step-12}
    }
    \animationframe{15}[
        Tidying up
    ]{
        \tikzfig{circuits/examples/reasoning/unfolding-dg/step-12}
    }{
        \tikzfig{circuits/examples/reasoning/unfolding-dg/step-13}
    }
\end{frame}

\begin{frame}
    \frametitle{Productivity}

    Any circuit has an \alert{instantaneous value} and a \alert{delayed subcircuit}.

    \[
        \tikzfig{circuits/components/circuits/f-seq}
        \,=\,
        \tikzfig{circuits/productivity/productive}    
    \]

    \wait
    These values are the elements of the corresponding stream!

\end{frame}

\begin{frame}
    \frametitle{Open circuits}

    We still cannot translate between \alert{open} circuits with the same behaviour.

    \wait

    \[
        \tikzfig{circuits/components/circuits/f-seq-unlabelled}
        \,=\,
        \tikzfig{circuits/examples/simple-and}    
        \quad
        \tikzfig{circuits/components/circuits/g-seq-unlabelled}
        \,=\,
        \tikzfig{circuits/examples/simple-or-nots}
    \]

    \wait

    When do two circuits have the \alert{same stream}?

\end{frame}

\begin{frame}
    \frametitle{Open circuits}

    We can think of circuits as \alert{state machines}:

    \[
        \tikzfig{circuits/components/circuits/f-seq}
        \wait\quad=\quad
        \tikzfig{circuits/full-abstraction/global-delay-f}
    \]

    \wait

    The circuit \(\tikzfig{circuits/components/circuits/f-pre-trace-unlabelled}\) produces the \alert{state transition} and \alert{output} of \(\tikzfig{circuits/components/circuits/f-seq}\).

    \wait

    \alert{Idea}: for all \alert{accessible states}, if the \alert{outputs} are equal then the \alert{original circuits} are equal under the equational theory.

    \tiny{(cf. Mealy machine bisimulation)}

\end{frame}

\begin{frame}
    \frametitle{Full abstraction}

    \begin{theorem}[]
        \(
            \tikzfig{circuits/components/circuits/f-seq} = \tikzfig{circuits/components/circuits/g-seq}
        \)
        if and only if their streams are equal.
    \end{theorem}
    \begin{proof}
        \visible<2->{
            \tikzfig{circuits/components/circuits/f-seq}
        }
        \visible<3->{
            \(=\) \tikzfig{circuits/full-abstraction/global-delay-f}
        }
        \visible<4->{
            \(=\)
        }
        \visible<3->{
            \tikzfig{circuits/full-abstraction/global-delay-g} \(=\)
        }
        \visible<2->{
            \tikzfig{circuits/components/circuits/g-seq}
        }        
    \end{proof}
\end{frame}

\begin{frame}
    \frametitle{Conclusion}

    We have presented a \alert{categorical framework} for sequential circuits

    \wait

    Circuits have semantics as \alert{stream functions}
    
    \wait

    It is easier to reason \alert{equationally}

    \wait

    We have \alert{full abstraction}: a correspondence between syntactic and semantic

    \wait

    Next step: refine the \alert{rewriting system}

\end{frame}