\section{Semantics}

\begin{frame}
    \frametitle{Interpretation}

    Values are interpreted in a \alert{lattice} \(\values\):

    \begin{minipage}{0.49\textwidth}
        \[
            \tikzfig{circuits/a4}
        \]
    \end{minipage}
    \begin{minipage}{0.49\textwidth}
        \begin{align*}
            \tikzfig{circuits/components/values/false} 
            \,&\mapsto\, 0 \\
            \tikzfig{circuits/components/values/true} 
            \,&\mapsto\, 1 \\
            \tikzfig{strings/structure/monoid/init} 
            \,&\mapsto\, \bot \\
            \tikzfig{strings/structure/monoid/init-white} 
            \,&\mapsto\, \top \\
        \end{align*}
    \end{minipage}

    Gates are interpreted as \alert{monotone functions} \(\morph{\overline{g}}{\valuetuple{m}}{\values}\).
\end{frame}

\begin{frame}
    \frametitle{Streams}

    The semantics of circuits is that of \alert{stream functions}.

    \pause

    A \alert{stream} \(\stream{\values}\) is an infinite sequence of values.
    
    \pause
    
    A \alert{stream function} \(\morph{f}{\stream{(\valuetuple{m})}}{\stream{(\valuetuple{n})}}\) consumes and produces streams.
\end{frame}

\begin{frame}
    \frametitle{Stream functions}

    Not all stream functions correspond to stream functions.

    \alert{Causal}: The \(i\)th element of a stream can only depend on the first \(i+1\) inputs.

    \alert{Monotone}: Computing the \(i\)th element of a stream is a monotone function.

    \alert{`Finite'}: After some prefix \(p\), computing the \(p+i\)th element of a stream is the same as computing the \(p+i+j\)th element.

    \begin{theorem}
        Every monotone causal `finite' stream function with corresponds to a class of sequential circuits. 
    \end{theorem}
\end{frame}